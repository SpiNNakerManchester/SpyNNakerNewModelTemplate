\documentclass[12pt]{article}

%\title{IF\_curr\_comb\_exp: combined exponential synaptic response}
\title{Implmentation of Synapse Models on SpiNNaker}
\author{Oliver Rhodes}
\date{\today}


\begin{document}
\maketitle

\section{Introduction}


\subsection{Instantaneous Rise and Exponential Decay}
\subsubsection*{Synaptic Response}


\subsubsection*{SpiNNaker Implementation}

\subsection{Alpha Response}
\subsubsection*{Synaptic Response}
The characteristic `alpha' response is defined by two coupled differential equations. Upon combination, and setting equal time constants, a response similar to combining two exponential responses is achieved. The response of $g_{syn}$, together with the two coupled differential equations, can be defined as:
\begin{eqnarray}
g_{syn}(t) & = & \bar{g}_{syn}fg(t) \nonumber \\
\frac{dg}{dt} & = & \frac{1}{\tau_{decay}} (-g + h) \nonumber \\
\frac{dh}{dt} & = & \frac{1}{\tau_{rise}}(-h) \nonumber
\label{eqn:alpha_coupled_diff_eqns}
\end{eqnarray}

Setting $\tau_{rise}=\tau_{decay} = \tau$ and solving for $h(t)$ via separation of variables:
\begin{eqnarray}
\int \frac{1}{h} dh & = & \int -\frac{1}{\tau}dt \nonumber \\
\ln{h} & = & -\frac{t}{\tau} + C \nonumber \\
h(t) & = & e^{-\frac{t}{\tau}+C} = ke^{-\frac{t}{\tau}} \nonumber
\label{eqn:solve_for_h}
\end{eqnarray}
Rearranging to the form $\frac{dy}{dt} +Py = Qt$, substituting for $h(t)$, and solving for $g(t)$ by integration via the particular integral (and the reverse product rule):
\begin{eqnarray}
\frac{dg}{dt} + \frac{g}{\tau} =\frac{k}{\tau}e^{-\frac{t}{\tau}} \nonumber \\
\mathrm{P.I.} =  e^{\int P dt} = e^{\frac{t}{\tau}} \nonumber \\
e^{\frac{t}{\tau}i}\frac{dg}{dt} +  e^{\frac{t}{\tau}\frac{g}{\tau}} = \frac{k}{\tau}e^{-\frac{t}{\tau}}e^{\frac{t}{\tau}} \nonumber \\
\frac{d}{dt} g e^{\frac{t}{\tau}} = \frac{k}{\tau}e^{-\frac{t}{\tau} + \frac{t}{\tau}} \nonumber \\
g e^{\frac{t}{\tau}} = \int \frac{k}{\tau}e^0 dt = \frac{kt}{\tau} \nonumber \\
g(t) = \frac{kt}{\tau}e^{-\frac{t}{\tau}}
\end{eqnarray}
\subsubsection*{SpiNNaker Implementation}


\subsection{Combined Exponentials}
\subsubsection*{Synaptic Response}
Upon the presynaptic neuron firing, a synapse response is triggered, causing some input to be delivered to the post-synaptic neuron. This document covers the implementation of a current response shaped via two exponential functions. First the target shape is introduced, followed by details of an implementation on the SpiNNaker platform.
\begin{eqnarray}
i(t) = f(e^{-\frac{t}{\tau_a}} - e^{-\frac{-t}{\tau_b}}) \\
t_{rise} = ln\frac{B\tau_x}{A\tau_x}(\frac{\tau_x\tau_x}{\tau_x - \tau_x} \\
f = \frac{1}{e^{-\frac{t_{rise}}{\tau_a}} - e^{-\frac{-t_{rise}}{\tau_b}}}
\label{eqn:combined_exponential_response}
\end{eqnarray}
\subsubsection*{SpiNNaker Implementation}
A generic combined exponential synapse is implemented to facilitate exploration of further synapse shapes defined through combinations of exopnential functions. The current $i(t)$ is therefore defined as in Eqn.\ref{eqn:generic_form}.
\begin{eqnarray}
i(t) = Ae^{\frac{-t}{\tau_a}} + Be^{\frac{-t}{\tau_b}}\\
\frac{di(t)}{dt} = \frac{-A}{\tau_a}e^{\frac{-t}{\tau_a} - \frac{B}{\tau_b}e^{\frac{-t}{\tau_b}}}
\label{eqn:generic_form}
\end{eqnarray}

To give the respone defined in Eqn.~\ref{eqn:combined_exponential_response}, coefficients are set $A=-B=f$. The response defined in ***cite*** can then be repeated by setting $\tau_a = \tau_{rise} = 1.7$ and $\tau_b = \tau_{decay} = 0.2$. These coefficients give the response shown in Fig.~\ref{fig:typical_response}.
\begin{figure}
\caption{Synapse behaviour replicating ***cite***, achieved through combining exponetnials according to \ref{eqn:generic_form} and setting coefficients $A=-B=f$ (where $f$ is calculated according to Eqn.~\ref{eqn:combined_exponential_response}), and $\tau_a = 1.7$ and $\tau_b = 0.2$.}
\label{fig:typical_response}
\end{figure}
Note that Fig.~\ref{fig:typical_response} also shows the two components before combining: this shows that for the coefficients given, the early response is dominated by the contribution from the first component -- designated $\tau_{rise}$ -- with the subsequent response dominated by the second component -- designated $\tau_{decay}$.



\subsection{Biphasic}
\subsubsection*{Synaptic Response}
\subsubsection*{SpiNNaker Implementation}

\section{Results}

\section{Discussion}

\end{document}
